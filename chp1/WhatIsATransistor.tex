\documentclass{article}
\usepackage{amsthm}
\usepackage{graphicx}

\begin{document}
\section{High Level Transistor}
We begin by exploiting the dynamics of the electron to explain how transistors work at the barebones level.
In essence, a transistor at its most fundamental level is a \textbf{switch} of some small current to a much largest current flow in separate part of the transistor. This ultimately allows us to denote the transistor in two states: 0 or 1. If the transistor is in the 0 state, there is no current outflowing and similarly if it is a 1 state, there is current outflowing from the transistor. 
\section{The Semiconductor Revelation}
At its core, a transistor attempts to drain charge(transfer electrons) from position to another effectively making the recieving position create a signal. However, this is not as easy as placing one positively charge atom near a negatively charge one. Instead, transistors capitalize on the unique properties of semi-conductors. A semi-conductor is a unique class of materials in which it is not a conductor or a insulator and therefore electrons don't necessarily flow through semi-conductors but also, we don't stop the flow of electron. For example, a penny-made of copper which is a metal- is a conductor meanwhile plastic is an insulator as current cannot flow through it. However, an element such as silicon behaves in different ways depending on how we dope it. Doping is a process in which we add impure elements to silicon to alter its electronic properties. Silicon is used for two unique reasons. The first is that it is a very common resource in our world as it can be found in sand or quartz. Secondly, it contains 4 valence electrons on its outer shell. As such, it can be doped with 5 or 3 valence electron elements to alter its properties. This leads us to discuss the two (simple) types of doping for silicon based transistors. The first is called n-type doping for negative doping and it involves integrating arsenic or phosprous with silicon. Since these elements have 5 valence electrons, the silicon will fill its outer valence shell and be left with a single negatively charged particle which will be free to move allowing electronic current to flow through the metal. P-type doping is not as simple. Silicon is combined with 3 valence electron elements such as boron or gallium. In this case, these elements form electron holes in the valence band of the silicon. These electron holes are now free to accept excess electrons from the n-doped silicon and this will allow us to create a movement of charge (See figure 1). 
\begin{figure}
\includegraphics[scale=0.5]{electronhole.png}
\end{figure}
\section{Our First Transistor Model}
We can now use the p and n doped silicon materials to build a directional charge and as such our first transistor. Consider we make a sandwhich of materials such that we have a n-p-n organization where we have a slice of p-type silicon in between two n-type pieces. Additionally, we add an electron source on end of the n-type silicon and an electron drain on the other end so we can apply and accept current through our system. Furthermore, let us add a conductor to the p-type silicon as well that will act as our switch. Recall that the n-type pieces both have an extra free electron while the p-type silicon contains an electron hole. When we do not apply any current, the hole accepts the free electron and we do not have any movement of electrons as the system reaches an equilibrium. However, if we add a small current into the conductor that is attached to the p-type silicon, we then induce movement of the electron holes in the p-type material which then allows electrons to be pulled from the emitter into the collector creating a current flow that is generally larger than the original base current. As such, we then have changed our transistor from a state of no current to a state of flowing current or from 0 to 1. This type of transistor is known as a junction transistor and is the simplest form of what has now become the building blocks of the modern internet. 
\begin{figure}
  \includegraphics[scale=0.2]{maxresdefault.jpg}
\end{figure}
\end{document}
